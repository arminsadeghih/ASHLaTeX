\label{chekide}
\chapter*{چکیده}
\addcontentsline{toc}{chapter}{چکیده}
مساله کوشی در ریاضیات بیشتر از نیم قرن قبل از تولد نظریه نسبیت عام مطرح بود و نقطه شروعی برای بررسی مساله کوشی در نسبیت عام و حتی ایده ای بود برای جدا کردن ابرسطح فضاگونه از 4 بعد فضا زمان یا به عبارت دیگر تجزیه $ADM$ بود که 4 دهه بعد از مطرح شدن کنش هیلبرت اینشتین ارائه شد. متریک القایی  و تانسور انحنای عرضی که پیشتر در ریاضیات با نام فرم بنیادی اول و دوم شناخته می شدند، به عنوان ستون های یکی از مهم ترین خصوصیت هایی که لازمه ی یک نظریه در فیزیک یعنی وجود و یکتایی جواب، مورد مطالعه قرار گرفتند. مساله کوشی در نسبیت عام تاثیر غیر مستقیم در پیشبرد این نظریه نیز داشت و آن شرایط مرزی در رهیافت غیرهموردا بود. بررسی شرایط مرزی در نسبیت عام در اوایل شکل گیری این نظریه غیرضروری به نظر می رسید ولی دقت بیشتر در مورد کروشه های پواسون، ضرورت بررسی این امر را اجتناب ناپذیر می کرد. سازگار بودن نسبیت عام با شرایط مرزی دیریشتله در 4 بعد و احتمال اعمال شرایط مرزی نویمان در بیشتر از 4 بعد این مساله را جالب توجه می کند.\\
در این پایان نامه علاوه بر مباحث گفته شده، معادلات حرکت نسبیت عام با استفاده از رهیافت هموردا نیز بررسی می شوند که پایه ی ارتباط جملات درجه دو ( که به معادلات حرکت ختم می شوند) با جملات سطحی یا مبحث هولوگرافی است. علاوه بر آن دو نظریه جایگزین بسیار مهم در گرانش یعنی نظریه اسکالر تانسوری و گرانش $f(R)$ و ارتباط این دو نظریه در فرمولبندی لاگرانژی و هامیلتونی در چارچوب های اینشتین و جوردن و همچنین هولوگرافی در گرانش $f(R)$ مورد بحث قرار خواهند گرفت.\\
کلید واژه ها: مساله کوشی، شرایط مرزی، هولوگرافی، گرانش $f(R)$، نظریه اسکالر تانسوری