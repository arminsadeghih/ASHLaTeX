% !TeX root=main.tex
% در این فایل، عنوان پایان‌نامه، مشخصات خود، متن تقدیمی‌، ستایش، سپاس‌گزاری و چکیده پایان‌نامه را به فارسی، وارد کنید.
% توجه داشته باشید که جدول حاوی مشخصات پروژه/پایان‌نامه/رساله و همچنین، مشخصات داخل آن، به طور خودکار، درج می‌شود.
%%%%%%%%%%%%%%%%%%%%%%%%%%%%%%%%%%%%
% دانشگاه خود را وارد کنید
\university{دانشگاه تهران}
% دانشکده، آموزشکده و یا پژوهشکده  خود را وارد کنید
\faculty{دانشگاه تهران\\دانشکدگان علوم\\ دانشکده فیزیک}
% گروه آموزشی خود را وارد کنید
\department{دانشکده فیزیک}
% گروه آموزشی خود را وارد کنید
\subject{فیزیک}
% گرایش خود را وارد کنید
\field{گرانش و کیهان‌شناسی}
% عنوان پایان‌نامه را وارد کنید
\title{عنوان}
% نام استاد(ان) راهنما را وارد کنید
\firstsupervisor{دکتر فاطمه شجاعی باغینی}
%\secondsupervisor{استاد راهنمای دوم}
% نام استاد(دان) مشاور را وارد کنید. چنانچه استاد مشاور ندارید، دستور پایین را غیرفعال کنید.
%\firstadvisor{استاد مشاور اول}
%\secondadvisor{استاد مشاور دوم}
% نام دانشجو را وارد کنید
\name{آرمین صادقی حسنوند}
% نام خانوادگی دانشجو را وارد کنید
\surname{}
% شماره دانشجویی دانشجو را وارد کنید
\studentID{}
% تاریخ پایان‌نامه را وارد کنید
\thesisdate{شهریور 1404}
% به صورت پیش‌فرض برای پایان‌نامه‌های کارشناسی تا دکترا به ترتیب از عبارات «پروژه»، «پایان‌نامه» و »رساله» استفاده می‌شود؛ اگر  نمی‌پسندید هر عنوانی را که مایلید در دستور زیر قرار داده و آنرا از حالت توضیح خارج کنید.
%\projectLabel{پایان‌نامه}

% به صورت پیش‌فرض برای عناوین مقاطع تحصیلی کارشناسی تا دکترا به ترتیب از عبارات «کارشناسی»، «کارشناسی ارشد» و »دکترا» استفاده می‌شود؛ اگر  نمی‌پسندید هر عنوانی را که مایلید در دستور زیر قرار داده و آنرا از حالت توضیح خارج کنید.

% صفحه بسم الله الرحمن الرحیم با یک فایل besm.jpg وارد شده است. چنانچه میخواهید آنرا تغییر دهید به خط 171 فایل ASH-UT-Thesis.cls مراجعه کنید
\besmPage
\firstPage
\newpage
%\davaranPage

% در این قسمت اسامی اساتید راهنما، مشاور و داور باید به صورت دستی وارد شوندو
% برای دانشکده فیزیک برگه اساتید جداگانه از آموزش دریافت می شود، در صورت نیاز استفاده کنید.
%\renewcommand{\arraystretch}{1.2}
%\begin{center}
%\begin{tabular}{| p{8mm} | p{18mm} | p{.17\textwidth} |p{14mm}|p{.2\textwidth}|c|}
%\hline
%ردیف	& سمت & نام و نام خانوادگی & مرتبه \newline دانشگاهی &	دانشگاه یا مؤسسه &	امضـــــــــــــا\\
%\hline
%۱ &	استاد راهنما				 & دکتر \newline  نام استاد & مرتبه علمی & دانشگاه \newline نام دانشگاه &  \\
%\hline
%۲ &  استاد مشاور				 & دکتر \newline  نام استاد & مرتبه علمی & دانشگاه \newline  نام دانشگاه & \\
%\hline
%۲ &  استاد مدعو\newline  خارجی			 & دکتر \newline نام استاد  & مرتبه علمی & دانشگاه \newline  نام دانشگاه & \\
%\hline
%۴ &	استاد مدعو\newline  خارجی			 & دکتر \newline نام استاد & مرتبه علمی & دانشگاه \newline  نام دانشگاه & \\
%\hline
%۳ &	استاد مدعو\newline  داخلی			 & دکتر\newline نام استاد & مرتبه علمی & دانشگاه \newline   نام دانشگاه & \\
%\hline
%۶ &	استاد مدعو\newline  داخلی			 & دکتر\newline نام استاد & مرتبه علمی & دانشگاه \newline   نام دانشگاه & \\
%\hline
%۷ &	استاد مدعو\newline  داخلی			 &دکتر \newline منام استاد & مرتبه علمی & دانشگاه \newline   نام دانشگاه & \\
%\hline
%\end{tabular}
%\end{center}
%\esalatPage
%\mojavezPage


% در اینجا فایل مربوط به گواهی دفاع که درجه ارزیابی دفاع شما در آن درج می شود (بعد از دفاع از آموزش در خواست کنید) را بصورت عکس یا پی دی اف در پوشه اصلی بریزید و نام آنرا در قسمت مربوطه بنویسید (  همه ی علامت های % را بردارید)
%\newpage
%\pagestyle{plain}
%\pagenumbering{harfi}
%‎\includegraphics[scale=0.85]{نام فایل}‎


% چنانچه مایل به چاپ صفحات «تقدیم»، «نیایش» و «سپاس‌گزاری» در خروجی نیستید، خط‌های زیر را با گذاشتن ٪  در ابتدای آنها غیرفعال کنید.
 % پایان‌نامه خود را تقدیم کنید!
\vspace{4cm}

{\nastaliq
{\Large
تقدیم به 

\quad 
پدر، مادر و خواهر عزیزم
\vspace{1.5cm}

\newdimen\xa
\xa=\textwidth
\advance \xa by -11cm
\hspace{\xa}
% نام کسانی که میخواهید تقدیم کنید
}}



%%%%%%%%%%%%%%%%%%%%%%%%%%%%%%%%%%%%
% کلمات کلیدی پایان‌نامه را درون {} وارد کنید

\keywords{کلمات کلیدی}
%چکیده پایان‌نامه را وارد کنید، برای ایجاد پاراگراف جدید از \\ استفاده کنید. اگر خط خالی دشته باشید، خطا خواهید گرفت.

\fa-abstract{چکیده
}

\abstractPage


\begin{acknowledgementpage}
% این قسمت، قسمت تشکر و قدردانی است به سلیقه خود بنویسید یک نمونه هم در زیر مشاهده می کنید.

وظیفه خود می‌دانم از زحمات بی‌دریغ استاد راهنمایم، سرکار خانم دکتر فاطمه شجاعی باغینی، صمیمانه قدردانی کنم. بی‌تردید بدون راهنمایی‌های ارزشمند ایشان، انجام این رساله میسر نمی‌شد. از دوستان عزیزم که در دوران دکتری، با حضور دلگرم‌کننده‌شان در این مسیر به من انگیزه دادند، سپاسگزارم. از داوران دفاع دکتری بنده که با نکات و پیشنهادهایشان تز دکتری من را بهبود بخشیدند کمال امتنان را دارم.

 در پایان، بوسه می‌زنم بر دستان  پدر و مادر عزیزم و از خانواده عزیزم به پاس عاطفه سرشار و گرمای امیدبخش وجودشان، که بهترین پشتیبان من بودند، تشکر می کنم.

% با استفاده از دستور زیر، امضای شما، به طور خودکار، درج می‌شود.
\signature 
\end{acknowledgementpage}


\newpage\clearpage