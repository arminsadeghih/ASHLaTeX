
%!TEX program = xelatex
% اگر قصد نوشتن پروژه کارشناسی را دارید، در خط زیر به جای msc، کلمه bsc و اگر قصد نوشتن پروژه دکترا را دارید، کلمه phd را قرار دهید. کلیه تنظیمات لازم، به طور خودکار، اعمال می‌شود.

% اگر مایلید پایان‌نامه شما دورو باشد به جای oneside  در خط زیر از twoside استفاده کنید
\documentclass[oneside,openany,phd]{ASH-UT-Thesis}

% مشخصات پایان‌نامه را در فایلهای faTitle و enTitle وارد نمایید.

% فایل commands.tex را مطالعه کنید؛ چون دستورات مربوط به فراخوانی بسته زی‌پرشین و دیگر بسته‌ها و ... در این فایل قرار دارد و بهتر   است که با نحوه استفاده از آنها آشنا شوید.
	\input{commands}

\begin{document}
	
\numberwithin{equation}{section}
% اعداد صفحات
\pagenumbering{harfi}
% صفحه عنوان. این صفحه به طور پیش فرض با قالب مورد نظر دانشکده فیزیک طراحی شده است. برای وارد کردن نام و مشخصات خود به فایل fatitle.tex مراجعه کنید.
% !TeX root=main.tex
% در این فایل، عنوان پایان‌نامه، مشخصات خود، متن تقدیمی‌، ستایش، سپاس‌گزاری و چکیده پایان‌نامه را به فارسی، وارد کنید.
% توجه داشته باشید که جدول حاوی مشخصات پروژه/پایان‌نامه/رساله و همچنین، مشخصات داخل آن، به طور خودکار، درج می‌شود.
%%%%%%%%%%%%%%%%%%%%%%%%%%%%%%%%%%%%
% دانشگاه خود را وارد کنید
\university{دانشگاه تهران}
% دانشکده، آموزشکده و یا پژوهشکده  خود را وارد کنید
\faculty{دانشگاه تهران\\دانشکدگان علوم\\ دانشکده فیزیک}
% گروه آموزشی خود را وارد کنید
\department{دانشکده فیزیک}
% گروه آموزشی خود را وارد کنید
\subject{فیزیک}
% گرایش خود را وارد کنید
\field{گرانش و کیهان‌شناسی}
% عنوان پایان‌نامه را وارد کنید
\title{مقلد‌های سیاه‌چاله}
% نام استاد(ان) راهنما را وارد کنید
\firstsupervisor{دکتر فاطمه شجاعی باغینی}
%\secondsupervisor{استاد راهنمای دوم}
% نام استاد(دان) مشاور را وارد کنید. چنانچه استاد مشاور ندارید، دستور پایین را غیرفعال کنید.
%\firstadvisor{استاد مشاور اول}
%\secondadvisor{استاد مشاور دوم}
% نام دانشجو را وارد کنید
\name{آرمین صادقی حسنوند}
% نام خانوادگی دانشجو را وارد کنید
\surname{}
% شماره دانشجویی دانشجو را وارد کنید
\studentID{}
% تاریخ پایان‌نامه را وارد کنید
\thesisdate{شهریور 1404}
% به صورت پیش‌فرض برای پایان‌نامه‌های کارشناسی تا دکترا به ترتیب از عبارات «پروژه»، «پایان‌نامه» و »رساله» استفاده می‌شود؛ اگر  نمی‌پسندید هر عنوانی را که مایلید در دستور زیر قرار داده و آنرا از حالت توضیح خارج کنید.
%\projectLabel{پایان‌نامه}

% به صورت پیش‌فرض برای عناوین مقاطع تحصیلی کارشناسی تا دکترا به ترتیب از عبارات «کارشناسی»، «کارشناسی ارشد» و »دکترا» استفاده می‌شود؛ اگر  نمی‌پسندید هر عنوانی را که مایلید در دستور زیر قرار داده و آنرا از حالت توضیح خارج کنید.

% صفحه بسم الله الرحمن الرحیم با یک فایل besm.jpg وارد شده است. چنانچه میخواهید آنرا تغییر دهید به خط 171 فایل ASH-UT-Thesis.cls مراجعه کنید
\besmPage
\firstPage
\setcounter{page}{1}
\newpage
%\davaranPage

% در این قسمت اسامی اساتید راهنما، مشاور و داور باید به صورت دستی وارد شوندو
% برای دانشکده فیزیک برگه اساتید جداگانه از آموزش دریافت می شود، در صورت نیاز استفاده کنید.
%\renewcommand{\arraystretch}{1.2}
%\begin{center}
%\begin{tabular}{| p{8mm} | p{18mm} | p{.17\textwidth} |p{14mm}|p{.2\textwidth}|c|}
%\hline
%ردیف	& سمت & نام و نام خانوادگی & مرتبه \newline دانشگاهی &	دانشگاه یا مؤسسه &	امضـــــــــــــا\\
%\hline
%۱ &	استاد راهنما				 & دکتر \newline  نام استاد & مرتبه علمی & دانشگاه \newline نام دانشگاه &  \\
%\hline
%۲ &  استاد مشاور				 & دکتر \newline  نام استاد & مرتبه علمی & دانشگاه \newline  نام دانشگاه & \\
%\hline
%۲ &  استاد مدعو\newline  خارجی			 & دکتر \newline نام استاد  & مرتبه علمی & دانشگاه \newline  نام دانشگاه & \\
%\hline
%۴ &	استاد مدعو\newline  خارجی			 & دکتر \newline نام استاد & مرتبه علمی & دانشگاه \newline  نام دانشگاه & \\
%\hline
%۳ &	استاد مدعو\newline  داخلی			 & دکتر\newline نام استاد & مرتبه علمی & دانشگاه \newline   نام دانشگاه & \\
%\hline
%۶ &	استاد مدعو\newline  داخلی			 & دکتر\newline نام استاد & مرتبه علمی & دانشگاه \newline   نام دانشگاه & \\
%\hline
%۷ &	استاد مدعو\newline  داخلی			 &دکتر \newline منام استاد & مرتبه علمی & دانشگاه \newline   نام دانشگاه & \\
%\hline
%\end{tabular}
%\end{center}
\esalatPage
\newpage
%\mojavezPage


% در اینجا فایل مربوط به گواهی دفاع که درجه ارزیابی دفاع شما در آن درج می شود (بعد از دفاع از آموزش در خواست کنید) را بصورت عکس یا پی دی اف در پوشه اصلی بریزید و نام آنرا در قسمت مربوطه بنویسید (  همه ی علامت های % را بردارید)
%\newpage
%\pagestyle{plain}
%\pagenumbering{harfi}
%‎\includegraphics[scale=0.85]{نام فایل}‎


% چنانچه مایل به چاپ صفحات «تقدیم»، «نیایش» و «سپاس‌گزاری» در خروجی نیستید، خط‌های زیر را با گذاشتن ٪  در ابتدای آنها غیرفعال کنید.
 % پایان‌نامه خود را تقدیم کنید!
\vspace{4cm}

{\nastaliq
{\Large
تقدیم به 

\quad 
پدر، مادر و خواهر عزیزم
\vspace{1.5cm}

\newdimen\xa
\xa=\textwidth
\advance \xa by -11cm
\hspace{\xa}
% نام کسانی که میخواهید تقدیم کنید
}}



%%%%%%%%%%%%%%%%%%%%%%%%%%%%%%%%%%%%
% کلمات کلیدی پایان‌نامه را درون {} وارد کنید

\keywords{سیاه‌چاله‌های منظم، اجسام فوق‌فشرده‌، تکینگی، گرانش سطح، افق، ترمودینامیک سیاه‌چاله، کره‌ی فوتون، سایه‌ی سیاه‌چاله‌، گرانش گاوس-بونه}
%چکیده پایان‌نامه را وارد کنید، برای ایجاد پاراگراف جدید از \\ استفاده کنید. اگر خط خالی دشته باشید، خطا خواهید گرفت.

\fa-abstract{
مدل‌های سیاه‌چاله‌ای، با وجود موفقیت‌های چشمگیر، پرسش‌هایی بنیادی در مورد ماهیت فضازمان، افق‌ها و تکینگی‌ها مطرح می‌کند. تکینگی‌ها در سیاه‌چاله‌ها ناکارآمدی نظریه را در آن محل نشان می‌دهد و این مشکلی است که باید حل شود. در میان سناریوهای ممکن برای رفع تکینگی، سیاهچاله‌های منظم و یا ستارگان فوق‌فشرده‌ی بدون افق مورد توجه قرار گرفته‌اند. دسته‌ی نخست، پاسخ‌هایی هستند که دست‌کم دارای یک افق بیرونی و یک افق درونی‌اند اما فاقد هسته‌ی تکین می‌باشند. دسته‌ی دوم پیکربندی‌های هستند که در آن‌ها هیچ افقی وجود ندارد و سطح ستاره درون کره‌ی فوتونی قرار دارد.
این دو دسته، مثال‌های اصلی مقلدهای سیاه‌چاله هستند. اجرام ناتکینی که نمی‌توان الزاماً آن‌ها را در رصد از سیاه‌چاله تشخیص داد، ضمن آنکه چالش‌های مطرح شده در مورد سیاه‌چاله‌ها را ندارد.
 رصد‌ها برای مشاهده‌ی اجسام پرجرم مانند سیاه‌چاله‌های کلان‌جرم به مشاهده‌ی سایه و امواج گرانشی محدود می‌شوند و  پیشرفت‌های اخیر در اخترشناسی امواج گرانشی و تصویربرداری الکترومغناطیسی در مقیاس افق، مسیرهای جدیدی برای آزمودن مدل‌های سیاه‌چاله‌ای و مقلدها گشوده‌اند و بنابراین  بررسی این جایگزین‌های سیاه‌چاله را به موضوع روز تبدیل کرده است. در میان این مقلدها، این رساله به دسته‌ی سیاه‌چاله‌های منظم و فرآیند تشکیل آن‌ها از رمبش یک ستاره می‌پردازد. ابتدا فرآیند رمبش به سیاه‌چاله‌های منظم شناخته‌شده‌ی باردین و هیوارد را بررسی می‌کنیم. سپس، این سناریو را به یک خانواده از سیاه‌چاله‌های منظم با هسته‌ی دوسیته تعمیم خواهیم داد که مدل‌های باردین و هیوارد حالات خاصی از آن هستند. همچنین سناریوی رمبش را برای متریک سیاه‌چاله‌ی منظم گرانش اینشتین-گاوس-بونه در چهار بعد، را بررسی خواهیم کرد و این پرسش را پاسخ می‌دهیم که آیا این متریک واقعا منظم است و آیا می تواند نقش یک مقلد را بازی کند یا خیر. پس از آن گرانش سطح برای ستاره‌ی در حال رمبش به سیاه‌چاله‌‌های منظم می‌پردازیم. ما این خانواده از سیاه‌چاله‌های منظم را به ابعاد بالا گسترش خواهیم داد، تانسور انرژی-تکانه‌ی آن را معرفی و کره‌ی فوتون و سایه را محاسبه کرده و با رصد مقایسه خواهیم کرد. در نهایت،  ما ترمودینامیک، آنتروپی، ظرفیت گرمایی و پایداری را برای این دسته از سیاهچاله‌های منظم را بررسی خواهیم کرد.
}

\abstractPage


\begin{acknowledgementpage}
% این قسمت، قسمت تشکر و قدردانی است به سلیقه خود بنویسید یک نمونه هم در زیر مشاهده می کنید.

وظیفه خود می‌دانم از زحمات بی‌دریغ استاد راهنمایم، سرکار خانم دکتر فاطمه شجاعی باغینی، صمیمانه قدردانی کنم. بی‌تردید بدون راهنمایی‌های ارزشمند ایشان، انجام این رساله میسر نمی‌شد. از دوستان عزیزم که در دوران دکتری، با حضور دلگرم‌کننده‌شان در این مسیر به من انگیزه دادند، سپاسگزارم. از داوران دفاع دکتری بنده که با نکات و پیشنهادهایشان تز دکتری من را بهبود بخشیدند کمال امتنان را دارم.

 در پایان، بوسه می‌زنم بر دستان  پدر و مادر عزیزم و از خانواده عزیزم به پاس عاطفه سرشار و گرمای امیدبخش وجودشان، که بهترین پشتیبان من بودند، تشکر می کنم.

% با استفاده از دستور زیر، امضای شما، به طور خودکار، درج می‌شود.
\signature 
\end{acknowledgementpage}


\newpage\clearpage
% فهرست
\tableofcontents
\cleardoublepage
\newpage
% چنانچه مایل هستید فهرست تصاویر در  فهرست قرار گیرد علامت % را بردارید.
%\listoffigures \newpage
% چنانچه مایل هستید فهرست جداول در فهرست قرار گیرد علامت % را بردارید.
%\listoftables  \newpage
%\addcontentsline{toc}{chapter}{\listalgorithmname}

% فایل مربوط به نمادگذاری، چنانچه در مقاله نمادگذاری ندارید یک علامت % پشت آن بگذارید. ( برای مثال مورد استفاده، در گرانش قبل از شروع مقاله باید اعلام کنید نمادگذاری شما +++- است یا ---+)
% سه خط اول فایل را پاک نکنید.
\label{gharardad}
\chapter*{قرارداد ها و نماد گذاری ها}
\addcontentsline{toc}{chapter}{قرارداد ها و نماد گذاری ها}
 قراردادها...
\pagestyle{fancy}

\pagenumbering{arabic}
\markboth{مقدمه}{}
\chapter*{مقدمه}
\label{moghadame}
\addcontentsline{toc}{chapter}{مقدمه}
\chapter{فصل اول}
\label{Chapdef}

فصل ۱ \cite{Woodard:2015zca}
\chapter{فصل دوم}
\label{ChapMim}
\thispagestyle{empty}
\chapter{فصل سوم}
\label{chapreg}
\chapter{فصل چهارم}
\label{ChapEGB}
\thispagestyle{empty}
%\include{Chapter5}
%\include{Chapter6}
%\include{Chapter7}
%\include{Chapter8}
% چنانچه مایل هستید فصل های بیشتری اضافه کنید، کافیست مانند بالا دستور include را اضافه کنید و نام فایل تک را در کروشه وارد کنید.
% مراجع
% دستوری برای تغییر نام کلمه «کتاب‌نامه» به «مراجع»
\renewcommand{\bibname}{مراجع}
\pagestyle{empty}
{
\onehalfspacing
\bibliographystyle{unsrt-fa-custom.bst}%{chicago-fa}%{plainnat-fa}%
\bibliography{References}
}

%\pagestyle{fancy}
%اگر از فایل تک برای ارجاع استفاده میکنید بجای قسمت بالا از کد پایین استفاده کنید
%\include{references}

	\appendix                        
% فصلهای پس از این قسمت به عنوان ضمیمه خواهند آمد. فرمت دانشکده فیزیک به این صورت است که پیوست پس از مراجع باید باشد، چنانچه می خواهید جای آن ها را عوض کنید جای دستورات با بالا عوض کنید.

\include{appendix1}
%% !TeX root=main.tex
% دستورات زیر باید در اولین فایل پیوست باشند. آنها را حذف نکنید!
\addtocontents{toc}{
    \protect\renewcommand\protect\cftchappresnum{\appendixname~}%
%    \protect\setlength{\cftchapnumwidth}{\mylenapp}}%
    
\chapter{عنوان پیوست}
\label{appendix2}
\thispagestyle{empty}

\baselineskip=.75cm
\onehalfspacing
% چنانچه مایل هستید پایان نامه شامل واژه نامه فارسی به انگلیسی باشد علامت % را بردارید. (معمولا نیازی نیست)
%\chapter*{واژه‌نامه فارسی به انگلیسی}\markboth{واژه‌نامه فارسی به انگلیسی}{واژه‌نامه فارسی به انگلیسی}
\addcontentsline{toc}{chapter}{واژه‌نامه فارسی به انگلیسی}
\thispagestyle{empty}

% کلماتی که میخواهید به همین فرمت زیر هم بنویسید.
\englishgloss{Christoffel}{همبسته}

% چنانچه مایل هستید پایان نامه شامل واژه نامه انگلیسی به فارسی باشد علامت % را بردارید. (معمولا نیازی نیست)
%\chapter*{واژه‌نامه  انگلیسی به  فارسی}\markboth{واژه‌نامه  انگلیسی به  فارسی}{واژه‌نامه  انگلیسی به  فارسی}
\addcontentsline{toc}{chapter}{واژه‌نامه  انگلیسی به  فارسی}
\thispagestyle{empty}
% کلماتی که میخواهید به همین فرمت زیر هم بنویسید.
\persiangloss{همبسته}{Christoffel}


\printindex
% !TeX root=main.tex
% در این فایل، عنوان پایان‌نامه، مشخصات خود و چکیده پایان‌نامه را به انگلیسی، وارد کنید.

%%%%%%%%%%%%%%%%%%%%%%%%%%%%%%%%%%%%
\baselineskip=.6cm
\begin{latin}
\latinuniversity{University of Tehran}
\latinfaculty{Faculty of Science\\ Department of Physics}
\latinsubject{Physics}
\latinfield{Gravitation and cosmology}
\latintitle{Title}
\firstlatinsupervisor{ِProf. ...}
%\secondlatinsupervisor{Second Supervisor}
%\firstlatinadvisor{First Advisor}
%\secondlatinadvisor{Second Advisor}
\latinname{Armin}
\latinsurname{Sadeghi}
\latinthesisdate{Shahrivar-1404}
\latinkeywords{Key words}
\en-abstract{
Abstract text...
}
\latinfirstPage
\end{latin}

\label{LastPage}

\end{document}