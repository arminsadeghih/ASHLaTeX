
%!TEX program = xelatex
% اگر قصد نوشتن پروژه کارشناسی را دارید، در خط زیر به جای msc، کلمه bsc و اگر قصد نوشتن پروژه دکترا را دارید، کلمه phd را قرار دهید. کلیه تنظیمات لازم، به طور خودکار، اعمال می‌شود.

% اگر مایلید پایان‌نامه شما دورو باشد به جای oneside  در خط زیر از twoside استفاده کنید
\documentclass[oneside,openany,phd]{ASH-UT-Thesis}

% مشخصات پایان‌نامه را در فایلهای faTitle و enTitle وارد نمایید.

% فایل commands.tex را مطالعه کنید؛ چون دستورات مربوط به فراخوانی بسته زی‌پرشین و دیگر بسته‌ها و ... در این فایل قرار دارد و بهتر   است که با نحوه استفاده از آنها آشنا شوید.
	% در این فایل، دستورها و تنظیمات مورد نیاز، آورده شده است.
%-------------------------------------------------------------------------------------------------------------------

% در ورژن جدید زی‌پرشین برای تایپ متن‌های ریاضی، این سه بسته، حتماً باید فراخوانی شود
\usepackage{amsthm,amssymb,amsmath}
% بسته‌ای برای تنطیم حاشیه‌های بالا، پایین، چپ و راست صفحه
\usepackage[top=35mm, bottom=25mm, left=25mm, right=35mm]{geometry}
% بسته‌‌ای برای ظاهر شدن شکل‌ها و تصاویر متن
\usepackage{graphicx}
% بسته‌ای برای رسم کادر
\usepackage{framed} 
% بسته‌‌ای برای چاپ شدن خودکار تعداد صفحات در صفحه «معرفی پایان‌نامه»
\usepackage{lastpage}
% بسته‌ و دستوراتی برای ایجاد لینک‌های رنگی با امکان جهش
\usepackage[pagebackref=false,colorlinks,linkcolor=blue,citecolor=blue]{hyperref}
% چنانچه قصد پرینت گرفتن نوشته خود را دارید، خط بالا را غیرفعال و  از دستور زیر استفاده کنید چون در صورت استفاده از دستور زیر‌‌، 
% لینک‌ها به رنگ سیاه ظاهر خواهند شد که برای پرینت گرفتن، مناسب‌تر است
%\usepackage[pagebackref=false]{hyperref}
% بسته‌ لازم برای تنظیم سربرگ‌ها
\usepackage[font={small}]{caption}
\usepackage{indentfirst}
\usepackage{fancyhdr}
\usepackage{setspace}
\usepackage{algorithm}
\usepackage{algorithmic}
\usepackage{subfigure}
\usepackage[subfigure]{tocloft}
\usepackage{yfonts}
\usepackage{mathrsfs}
\usepackage{hyperref}
\usepackage{amsmath}
\usepackage{relsize} 
\allowdisplaybreaks
\usepackage{esvect}
\usepackage{indentfirst}
\usepackage{placeins}
\usepackage{booktabs}
\usepackage{adjustbox}
\usepackage{hhline}
\usepackage{rotating}
\usepackage{multicol}
\usepackage{xparse}
\usepackage[compress]{cite}
\interfootnotelinepenalty=10000
% بسته‌ای برای ظاهر شدن «مراجع» و «نمایه» در فهرست مطالب
\usepackage[nottoc]{tocbibind}
% دستورات مربوط به ایجاد نمایه
\usepackage{makeidx}
\makeindex
%%%%%%%%%%%%%%%%%%%%%%%%%%
%دستور جاگزین کردن $$ و کوچک کردنآن

\newcommand{\m}[1]{\scalebox{0.9}{$\displaystyle #1$}}
%کوچک کردن عبارات ریاضی در align و equation
\everydisplay{\small}

%%%%%%%%%%%%%%%%%%%%%%%%%%
% فراخوانی بسته زی‌پرشین و تعریف قلم فارسی و انگلیسی

\usepackage{zref-perpage}
\zmakeperpage{footnote} 
\usepackage[T1]{fontenc}
\usepackage{xepersian}
\settextfont[Scale=1]{XB Niloofar}
\setlatintextfont[Scale=0.9]{Times New Roman} % Default LaTeX serif font
\usepackage{perpage}
\MakePerPage{footnote} 
\usepackage{bidiftnxtra}
% چنانچه مایل هستید اعداد پاورقی های با استفاده شده با \LTRfootnote به انگلیسی باشند،‌ خطوط زیر را حذف کنید.
\makeatletter
\bidi@patchcmd\@makefntext\@makefnmark{%
	\setpersianfont
	\@makefnmark
}{}{}
\makeatother
%%%%%%%%%%%%%%%%%%%%%%%%%%
% چنانچه می‌خواهید اعداد در فرمول‌ها، فارسی باشد، خط زیر را فعال کنید
%\setdigitfont[Scale=1]{XB Zar}%{Persian Modern}

%%%%%%%%%%%%%%%%%%%%%%%%%%
% تعریف قلم‌های فارسی و انگلیسی اضافی برای استفاده در بعضی از قسمت‌های متن
\defpersianfont\titlefont[Scale=1]{XB Titre}
\defpersianfont\iranic[Scale=1.1]{XB Zar Oblique}%{Italic}%
\defpersianfont\nastaliq[Scale=1.2]{IranNastaliq}
%%%%%%%%%%%%%%%%%%%%%%%%%%
%\renewcommand{\abstractname}{}
% دستوری برای حذف کلمه «abstract»
%\renewcommand{\latinabstract}{}
% دستوری برای تغییر نام کلمه «اثبات» به «برهان»
\renewcommand\proofname{\textbf{برهان}}
% دستوری برای تغییر نام کلمه «کتاب‌نامه» به «مراجع»
\renewcommand{\bibname}{مراجع}
% دستوری برای تعریف واژه‌نامه انگلیسی به فارسی
\newcommand\persiangloss[2]{#1\dotfill\lr{#2}\\}
% دستوری برای تعریف واژه‌نامه فارسی به انگلیسی 
\newcommand\englishgloss[2]{#2\dotfill\lr{#1}\\}
% تعریف دستور جدید «\پ» برای خلاصه‌نویسی جهت نوشتن عبارت «پروژه/پایان‌نامه/رساله»
\newcommand{\پ}{پروژه/پایان‌نامه/رساله }

%\newcommand\BackSlash{\char`\\}

%%%%%%%%%%%%%%%%%%%%%%%%%%
\SepMark{-}

% تعریف و نحوه ظاهر شدن عنوان قضیه‌ها، تعریف‌ها، مثال‌ها و ...
\theoremstyle{definition}
\newtheorem{definition}{تعریف}[section]
\newtheorem{theorem}[definition]{قضیه}
\newtheorem{lemma}[definition]{لم}
\newtheorem{proposition}[definition]{گزاره}
\newtheorem{corollary}[definition]{نتیجه}
\newtheorem{remark}[definition]{ملاحظه}
\theoremstyle{definition}
\newtheorem{example}[definition]{مثال}

%\renewcommand{\theequation}{\thechapter-\arabic{equation}}
%\def\bibname{مراجع}
\numberwithin{algorithm}{chapter}
%\def\listalgorithmname{فهرست الگوریتم‌ها}
\def\listfigurename{فهرست تصاویر}
\def\listtablename{فهرست جداول}

%%%%%%%%%%%%%%%%%%%%%%%%%%%%
% دستورهایی برای سفارشی کردن سربرگ صفحات
% \newcommand{\SetHeader}{
	% \csname@twosidetrue\endcsname
	% \pagestyle{fancy}
	% \fancyhf{} 
	% \fancyhead[OL,EL]{\thepage}
	% \fancyhead[OR]{\small\rightmark}
	% \fancyhead[ER]{\small\leftmark}
	% \renewcommand{\chaptermark}[1]{%
		% \markboth{\thechapter-\ #1}{}}
	% }
%%%%%%%%%%%%5
%\def\MATtextbaseline{1.5}
%\renewcommand{\baselinestretch}{\MATtextbaseline}
\doublespacing
%%%%%%%%%%%%%%%%%%%%%%%%%%%%%
% دستوراتی برای اضافه کردن کلمه «فصل» در فهرست مطالب

\newlength\mylenprt
\newlength\mylenchp
\newlength\mylenapp

\renewcommand\cftpartpresnum{\partname~}
\renewcommand\cftchappresnum{\chaptername~}
\renewcommand\cftchapaftersnum{:}

\settowidth\mylenprt{\cftpartfont\cftpartpresnum\cftpartaftersnum}
\settowidth\mylenchp{\cftchapfont\cftchappresnum\cftchapaftersnum}
\settowidth\mylenapp{\cftchapfont\appendixname~\cftchapaftersnum}
\addtolength\mylenprt{\cftpartnumwidth}
\addtolength\mylenchp{\cftchapnumwidth}
\addtolength\mylenapp{\cftchapnumwidth}

\setlength\cftpartnumwidth{\mylenprt}
\setlength\cftchapnumwidth{\mylenchp}	

\makeatletter
{\def\thebibliography#1{\chapter*{\refname\@mkboth
			{\uppercase{\refname}}{\uppercase{\refname}}}\list
		{[\arabic{enumi}]}{\settowidth\labelwidth{[#1]}
			\rightmargin\labelwidth
			\advance\rightmargin\labelsep
			\advance\rightmargin\bibindent
			\itemindent -\bibindent
			\listparindent \itemindent
			\parsep \z@
			\usecounter{enumi}}
		\def\newblock{}
		\sloppy
		\sfcode`\.=1000\relax}}
\makeatother





\begin{document}
	
\numberwithin{equation}{section}
% اعداد صفحات
\pagenumbering{harfi}
% صفحه عنوان. این صفحه به طور پیش فرض با قالب مورد نظر دانشکده فیزیک طراحی شده است. برای وارد کردن نام و مشخصات خود به فایل fatitle.tex مراجعه کنید.
% !TeX root=main.tex
% در این فایل، عنوان پایان‌نامه، مشخصات خود، متن تقدیمی‌، ستایش، سپاس‌گزاری و چکیده پایان‌نامه را به فارسی، وارد کنید.
% توجه داشته باشید که جدول حاوی مشخصات پروژه/پایان‌نامه/رساله و همچنین، مشخصات داخل آن، به طور خودکار، درج می‌شود.
%%%%%%%%%%%%%%%%%%%%%%%%%%%%%%%%%%%%
% دانشگاه خود را وارد کنید
\university{دانشگاه تهران}
% دانشکده، آموزشکده و یا پژوهشکده  خود را وارد کنید
\faculty{دانشگاه تهران\\دانشکدگان علوم\\ دانشکده فیزیک}
% گروه آموزشی خود را وارد کنید
\department{دانشکده فیزیک}
% گروه آموزشی خود را وارد کنید
\subject{فیزیک}
% گرایش خود را وارد کنید
\field{گرانش و کیهان‌شناسی}
% عنوان پایان‌نامه را وارد کنید
\title{عنوان}
% نام استاد(ان) راهنما را وارد کنید
\firstsupervisor{دکتر فاطمه شجاعی باغینی}
%\secondsupervisor{استاد راهنمای دوم}
% نام استاد(دان) مشاور را وارد کنید. چنانچه استاد مشاور ندارید، دستور پایین را غیرفعال کنید.
%\firstadvisor{استاد مشاور اول}
%\secondadvisor{استاد مشاور دوم}
% نام دانشجو را وارد کنید
\name{آرمین صادقی حسنوند}
% نام خانوادگی دانشجو را وارد کنید
\surname{}
% شماره دانشجویی دانشجو را وارد کنید
\studentID{}
% تاریخ پایان‌نامه را وارد کنید
\thesisdate{شهریور 1404}
% به صورت پیش‌فرض برای پایان‌نامه‌های کارشناسی تا دکترا به ترتیب از عبارات «پروژه»، «پایان‌نامه» و »رساله» استفاده می‌شود؛ اگر  نمی‌پسندید هر عنوانی را که مایلید در دستور زیر قرار داده و آنرا از حالت توضیح خارج کنید.
%\projectLabel{پایان‌نامه}

% به صورت پیش‌فرض برای عناوین مقاطع تحصیلی کارشناسی تا دکترا به ترتیب از عبارات «کارشناسی»، «کارشناسی ارشد» و »دکترا» استفاده می‌شود؛ اگر  نمی‌پسندید هر عنوانی را که مایلید در دستور زیر قرار داده و آنرا از حالت توضیح خارج کنید.

% صفحه بسم الله الرحمن الرحیم با یک فایل besm.jpg وارد شده است. چنانچه میخواهید آنرا تغییر دهید به خط 171 فایل ASH-UT-Thesis.cls مراجعه کنید
\besmPage
\firstPage
\newpage
%\davaranPage

% در این قسمت اسامی اساتید راهنما، مشاور و داور باید به صورت دستی وارد شوندو
% برای دانشکده فیزیک برگه اساتید جداگانه از آموزش دریافت می شود، در صورت نیاز استفاده کنید.
%\renewcommand{\arraystretch}{1.2}
%\begin{center}
%\begin{tabular}{| p{8mm} | p{18mm} | p{.17\textwidth} |p{14mm}|p{.2\textwidth}|c|}
%\hline
%ردیف	& سمت & نام و نام خانوادگی & مرتبه \newline دانشگاهی &	دانشگاه یا مؤسسه &	امضـــــــــــــا\\
%\hline
%۱ &	استاد راهنما				 & دکتر \newline  نام استاد & مرتبه علمی & دانشگاه \newline نام دانشگاه &  \\
%\hline
%۲ &  استاد مشاور				 & دکتر \newline  نام استاد & مرتبه علمی & دانشگاه \newline  نام دانشگاه & \\
%\hline
%۲ &  استاد مدعو\newline  خارجی			 & دکتر \newline نام استاد  & مرتبه علمی & دانشگاه \newline  نام دانشگاه & \\
%\hline
%۴ &	استاد مدعو\newline  خارجی			 & دکتر \newline نام استاد & مرتبه علمی & دانشگاه \newline  نام دانشگاه & \\
%\hline
%۳ &	استاد مدعو\newline  داخلی			 & دکتر\newline نام استاد & مرتبه علمی & دانشگاه \newline   نام دانشگاه & \\
%\hline
%۶ &	استاد مدعو\newline  داخلی			 & دکتر\newline نام استاد & مرتبه علمی & دانشگاه \newline   نام دانشگاه & \\
%\hline
%۷ &	استاد مدعو\newline  داخلی			 &دکتر \newline منام استاد & مرتبه علمی & دانشگاه \newline   نام دانشگاه & \\
%\hline
%\end{tabular}
%\end{center}
%\esalatPage
%\mojavezPage


% در اینجا فایل مربوط به گواهی دفاع که درجه ارزیابی دفاع شما در آن درج می شود (بعد از دفاع از آموزش در خواست کنید) را بصورت عکس یا پی دی اف در پوشه اصلی بریزید و نام آنرا در قسمت مربوطه بنویسید (  همه ی علامت های % را بردارید)
%\newpage
%\pagestyle{plain}
%\pagenumbering{harfi}
%‎\includegraphics[scale=0.85]{نام فایل}‎


% چنانچه مایل به چاپ صفحات «تقدیم»، «نیایش» و «سپاس‌گزاری» در خروجی نیستید، خط‌های زیر را با گذاشتن ٪  در ابتدای آنها غیرفعال کنید.
 % پایان‌نامه خود را تقدیم کنید!
\vspace{4cm}

{\nastaliq
{\Large
تقدیم به 

\quad 
پدر، مادر و خواهر عزیزم
\vspace{1.5cm}

\newdimen\xa
\xa=\textwidth
\advance \xa by -11cm
\hspace{\xa}
% نام کسانی که میخواهید تقدیم کنید
}}



%%%%%%%%%%%%%%%%%%%%%%%%%%%%%%%%%%%%
% کلمات کلیدی پایان‌نامه را درون {} وارد کنید

\keywords{کلمات کلیدی}
%چکیده پایان‌نامه را وارد کنید، برای ایجاد پاراگراف جدید از \\ استفاده کنید. اگر خط خالی دشته باشید، خطا خواهید گرفت.

\fa-abstract{چکیده
}

\abstractPage


\begin{acknowledgementpage}
% این قسمت، قسمت تشکر و قدردانی است به سلیقه خود بنویسید یک نمونه هم در زیر مشاهده می کنید.

وظیفه خود می‌دانم از زحمات بی‌دریغ استاد راهنمایم، سرکار خانم دکتر فاطمه شجاعی باغینی، صمیمانه قدردانی کنم. بی‌تردید بدون راهنمایی‌های ارزشمند ایشان، انجام این رساله میسر نمی‌شد. از دوستان عزیزم که در دوران دکتری، با حضور دلگرم‌کننده‌شان در این مسیر به من انگیزه دادند، سپاسگزارم. از داوران دفاع دکتری بنده که با نکات و پیشنهادهایشان تز دکتری من را بهبود بخشیدند کمال امتنان را دارم.

 در پایان، بوسه می‌زنم بر دستان  پدر و مادر عزیزم و از خانواده عزیزم به پاس عاطفه سرشار و گرمای امیدبخش وجودشان، که بهترین پشتیبان من بودند، تشکر می کنم.

% با استفاده از دستور زیر، امضای شما، به طور خودکار، درج می‌شود.
\signature 
\end{acknowledgementpage}


\newpage\clearpage
% فهرست
\tableofcontents
\cleardoublepage
\newpage
% چنانچه مایل هستید فهرست تصاویر در  فهرست قرار گیرد علامت % را بردارید.
%\listoffigures \newpage
% چنانچه مایل هستید فهرست جداول در فهرست قرار گیرد علامت % را بردارید.
%\listoftables  \newpage
%\addcontentsline{toc}{chapter}{\listalgorithmname}

% فایل مربوط به نمادگذاری، چنانچه در مقاله نمادگذاری ندارید یک علامت % پشت آن بگذارید. ( برای مثال مورد استفاده، در گرانش قبل از شروع مقاله باید اعلام کنید نمادگذاری شما +++- است یا ---+)
% سه خط اول فایل را پاک نکنید.
\label{gharardad}
\chapter*{قرارداد ها و نماد گذاری ها}
\addcontentsline{toc}{chapter}{قرارداد ها و نماد گذاری ها}
 قراردادها...
\pagestyle{fancy}
% دستور زیر باید در اول فایل شما باشد. آن را حذف نکنید!
\pagenumbering{arabic}
\markboth{مقدمه}{}
\chapter*{مقدمه}
\label{moghadame}
\addcontentsline{toc}{chapter}{مقدمه}
مقدمه...

\chapter{فصل اول}
\label{Chapdef}
\chapter{فصل دوم}
\label{ChapMim}
\thispagestyle{empty}
\chapter{فصل سوم}
\label{chapreg}
%\chapter{فصل چهارم}
\label{ChapEGB}
\thispagestyle{empty}
%\include{Chapter5}
%\include{Chapter6}
%\include{Chapter7}
%\include{Chapter8}
% چنانچه مایل هستید فصل های بیشتری اضافه کنید، کافیست مانند بالا دستور include را اضافه کنید و نام فایل تک را در کروشه وارد کنید.
% مراجع
% دستوری برای تغییر نام کلمه «کتاب‌نامه» به «مراجع»
\renewcommand{\bibname}{مراجع}
\pagestyle{empty}
{
\onehalfspacing
\bibliographystyle{unsrt-fa-custom.bst}%{chicago-fa}%{plainnat-fa}%
\bibliography{References}
}

%\pagestyle{fancy}
%اگر از فایل تک برای ارجاع استفاده میکنید بجای قسمت بالا از کد پایین استفاده کنید
%\include{references}

	\appendix                        
% فصلهای پس از این قسمت به عنوان ضمیمه خواهند آمد. فرمت دانشکده فیزیک به این صورت است که پیوست پس از مراجع باید باشد، چنانچه می خواهید جای آن ها را عوض کنید جای دستورات با بالا عوض کنید.

% !TeX root=main.tex
% دستورات زیر باید در اولین فایل پیوست باشند. آنها را حذف نکنید!
\addtocontents{toc}{
	\protect\renewcommand\protect\cftchappresnum{\appendixname~}%
	\protect\setlength{\cftchapnumwidth}{\mylenapp}}%

\chapter{پیوست}
\label{appendix1}
\thispagestyle{empty}
%% !TeX root=main.tex
% دستورات زیر باید در اولین فایل پیوست باشند. آنها را حذف نکنید!
\addtocontents{toc}{
    \protect\renewcommand\protect\cftchappresnum{\appendixname~}%
%    \protect\setlength{\cftchapnumwidth}{\mylenapp}}%
    
\chapter{عنوان پیوست}
\label{appendix2}
\thispagestyle{empty}

\baselineskip=.75cm
\onehalfspacing
% چنانچه مایل هستید پایان نامه شامل واژه نامه فارسی به انگلیسی باشد علامت % را بردارید. (معمولا نیازی نیست)
%\chapter*{واژه‌نامه فارسی به انگلیسی}\markboth{واژه‌نامه فارسی به انگلیسی}{واژه‌نامه فارسی به انگلیسی}
\addcontentsline{toc}{chapter}{واژه‌نامه فارسی به انگلیسی}
\thispagestyle{empty}

% کلماتی که میخواهید به همین فرمت زیر هم بنویسید.
\englishgloss{Christoffel}{همبسته}

% چنانچه مایل هستید پایان نامه شامل واژه نامه انگلیسی به فارسی باشد علامت % را بردارید. (معمولا نیازی نیست)
%\chapter*{واژه‌نامه  انگلیسی به  فارسی}\markboth{واژه‌نامه  انگلیسی به  فارسی}{واژه‌نامه  انگلیسی به  فارسی}
\addcontentsline{toc}{chapter}{واژه‌نامه  انگلیسی به  فارسی}
\thispagestyle{empty}
% کلماتی که میخواهید به همین فرمت زیر هم بنویسید.
\persiangloss{همبسته}{Christoffel}


\printindex
% !TeX root=main.tex
% در این فایل، عنوان پایان‌نامه، مشخصات خود و چکیده پایان‌نامه را به انگلیسی، وارد کنید.

%%%%%%%%%%%%%%%%%%%%%%%%%%%%%%%%%%%%
\baselineskip=.6cm
\begin{latin}
\latinuniversity{University of Tehran}
\latinfaculty{Faculty of Science\\ Department of Physics}
\latinsubject{Physics}
\latinfield{Gravitation and cosmology}
\latintitle{Black hole mimickers}
\firstlatinsupervisor{ِProf. Fatimah Shojai Baghini}
%\secondlatinsupervisor{Second Supervisor}
%\firstlatinadvisor{First Advisor}
%\secondlatinadvisor{Second Advisor}
\latinname{Armin}
\latinsurname{Sadeghi}
\latinthesisdate{Shahrivar-1404}
\latinkeywords{Regular black holes, Singularity, Ultracompact object, Surface gravity, Horizon, Black hole thermodynamics, Photon sphere, Shadow, Gauss-Bonnet gravity}
\en-abstract{
Black hole models, despite their remarkable successes, raise fundamental questions about the nature of spacetime, horizons, and singularities. The presence of singularities inside black holes suggests a theoretical breakdown that must be resolved. Black hole mimickers have been proposed as alternative candidates. Among the possible scenarios for addressing singularities, regular black holes and horizon-less ultra-compact stars have attracted attention. The former possess at least an outer and an inner horizon but lack a singular core. The latter are configurations without horizons and the stellar surface lies within the photon sphere. 
These two categories are the main examples of black hole mimickers. They are non-singular objects that cannot necessarily be distinguished from black holes through observation. They also avoid the challenges associated with black holes. 
Observations of massive objects such as supermassive black holes are limited to shadow measurements and gravitational waves. Recent advances in gravitational wave astronomy and horizon-scale electromagnetic imaging have opened new pathways to test these models. These advances have made the study of black holes and their mimicker models, an important topic. 
This thesis focuses on regular black holes  as a class of mimicker models and examines their formation through stellar collapse.
 We first examine the gravitational collapse leading to well-known  Bardeen and Hayward regular black holes.  Next, we generalize this scenario to a class of regular black holes with a de Sitter core. The Bardeen and Hayward solutions are two special cases of this class. Then, we study the dynamics of horizons during the collapse.
Additionally, we also analyze the collapse scenario for a regular black hole metric in four-dimensional Einstein–Gauss–Bonnet gravity to determine  whether this metric is truly regular and whether it can effectively serve as a mimicker. Furthermore, we study the surface gravity of collapsing stars that form regular black holes. We  extend this class of regular black holes to higher dimensions, introduce their energy–momentum tensor, analyze their photon spheres and shadows, and compare them with observations. Finally, we explore photon spheres, shadow, thermodynamics, entropy, heat capacity, and stability of this class of regular black holes.
}
\latinfirstPage
\end{latin}

\label{LastPage}

\end{document}